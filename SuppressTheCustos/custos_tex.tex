% !TEX TS-program = lualatex
% !TEX encoding = UTF-8
% greg_template

%greg_template_basic.tex
%This is a lualatex template fitted for use with greg.exe

%There are tokens like this:    Suppress the Custos
%These are special tokens that are either loaded from the settings in the gabc file
%or they are set per instance by the user in greg's main interface

%You can use your own custom tokens via greg's Token Editor

\documentclass[11pt]{article} % use larger type; default would be 10pt

% usual packages loading:
\usepackage{luatextra}
\usepackage{graphicx} % support the \includegraphics command and options
\usepackage{geometry} % See geometry.pdf to learn the layout options. There are lots.
\geometry{letterpaper} % or letterpaper (US) or a5paper or....
\usepackage{gregoriotex} % for gregorio score inclusion
\usepackage{fullpage} % to reduce the margins

% choose the language of the document here
\usepackage[latin]{babel}

% use the two following package for using normal TeX fonts
\usepackage[T1]{fontenc}
\usepackage[utf8]{luainputenc}

% If you use usual TeX fonts, here is a starting point:
\usepackage{times}
% to change the font to something better, you can install the kpfonts package (if not already installed). To do so
% go open the "TeX Live Manager" in the Menu Start->All Programs->TeX Live 2010

% here we begin the document
\begin{document}
\pagestyle{empty}		%removes page numbering


% The title:
\begin{center}\begin{huge}\textsc{Suppress the Custos}\end{huge}\end{center}

% Here we set the space around the initial.
% Please report to http://home.gna.org/gregorio/gregoriotex/details for more details and options
\setspaceafterinitial{2.2mm plus 0em minus 0em}
\setspacebeforeinitial{2.2mm plus 0em minus 0em}

% Here we set the initial font. Change 43 if you want a bigger initial.
\def\greinitialformat#1{  {\fontsize{28}{28}\selectfont #1}  }

% We set red lines here, comment it if you want black ones.



% We set VII above the initial.
\setfirstannotation{\fontsize{10}{10}{ }}%
\setsecondannotation{\sc{\fontsize{10}{10}{ }}}

% We type a text in the top right corner of the score:
\commentary{{\small \emph{ }}}

% and finally we include the score. The file must be in the same directory as this one.
\includescore{custos}

\end{document}
